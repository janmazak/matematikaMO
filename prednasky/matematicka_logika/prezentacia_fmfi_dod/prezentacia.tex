\documentclass[12pt]{beamer}
\usetheme{default}
\usecolortheme{crane}
%\usetheme{Madrid}

\usepackage[utf8x]{inputenc}
\usepackage[T1]{fontenc}
\usepackage[slovak]{babel}
\usepackage{ucs}
\usepackage{amsmath}
\usepackage{graphicx}
\usepackage{array}
\usepackage{amsmath, amssymb}
%\usepackage[inline]{asymptote}

%\setbeamersize{text margin left=1pt,text margin right=1pt}

\theoremstyle{definition}
\newtheorem{uloha}{Úloha}

\let\o=\vee
\let\a=\wedge
\let\bigo=\bigvee
\let\biga=\bigwedge
\let\A\forall
\let\E\exists

\begin{document}

\title{Ako matematická logika umožnila automatizované usudzovanie}
\author{Ján Mazák}
\institute{FMFI UK Bratislava}
\date{}
\frame{\titlepage}

\begin{frame}
%\frametitle{Čo je výrok?}
Čo je výrok? \emph{Oznamovacia veta},
\begin{itemize}
	\item ktorá má pravdivostnú hodnotu.
	\item ktorej sa dá priradiť pravdivostná hodnota.
	\item pre ktorú má zmysel otázka na jej platnosť, správnosť, pravdivosť.
\end{itemize}
\end{frame}

\begin{frame}
Nech $x$ je kladné reálne číslo. Potom $x^2 > 0$.
\end{frame}

\begin{frame}
Predstavme si, že $p$, $q$ sú výroky.
\begin{itemize}
	\item Je \uv{$p$ or not $p$} výrok?
	\item Je \uv{$p$ or not $q$} výrok?
\end{itemize}
\end{frame}

\begin{frame}
Ako definovať \emph{jednoduchý výrok}? \uv{Neobsahuje logické spojky.}
\begin{itemize}
	\item $x$ delí $y$
	\item existuje číslo $k$ také, že $k$ je prirodzené a $y = kx$
\end{itemize}
\vspace*{3mm}
Hovoríme to isté, lenže raz s logickou spojkou a raz bez.\\[6mm]

Pritom ak chceme uchopiť zložitosť výrokov, treba aspoň vedieť určiť, ktoré sú tie najjednoduchšie.
\end{frame}

\begin{frame}
Mimo Zeme žijú niekdajší obyvatelia Atlantídy.\\[5mm]
Počet hviezd je nepárny.
\end{frame}

\begin{frame}
Zajtra dva atómy vodíka vytvoria jadro hélia.\\[5mm]
\pause
Odteraz až navždy každú sekundu dva atómy vodíka vytvoria jadro hélia.
\end{frame}

\begin{frame}
Postupnosť $0,1,2,3,4,5,6,\dots$ je rastúca.\\[4mm]
\begin{itemize}
\item Je to výrok?
\pause
\item Ako vieme, ako bude postupnosť pokračovať?
\pause
\item Aký je význam troch bodiek ako symbolu? Môže byť súčasťou výroku popis algoritmu?
\pause
\item Môže byť výrok nekonečne dlhý?
\end{itemize}
\end{frame}

\begin{frame}
Bu bayonot emas.\\[4mm]
\begin{itemize}
\item Znamená \uv{toto nie je výrok} v uzbečtine. Aký jazyk je prípustný?
\item Čo ak má v rôznych jazykoch ten istý reťazec rôzny význam?
\pause
\item Môže výrok hovoriť o sebe?
\item Môže viesť k paradoxom: Zoberme množinu $X$ všetkých množín, ktoré neobsahujú samé seba.
Je veta \uv{$X$ patrí do $X$} výrok? Nemôže to byť ani pravda, ani nepravda,
pritom to vyzerá ako korektné matematické vyjadrenie.
\end{itemize}
\end{frame}

\begin{frame}
Chlieb predávajú v potravinách.\\[4mm]
\pause
\begin{itemize}
\item každý chlieb predávajú len v potravinách a nikde inde
\item každé potraviny predávajú aspoň jeden chlieb
\item existujú potraviny, ktoré predávajú aspoň jeden chlieb
\item existuje druh chleba, ktorý predávajú len v miestnej predajni potravín
\item existuje druh chleba, ktorý predávajú v každých potravinách
\item \dots
\end{itemize}
\end{frame}

\begin{frame}{Sumarizácia problémov}
\begin{itemize}
\item Pravdivosť je veľmi ťažké uchopiť.
\item Výrok v sebe nesie odkazy na reálny svet, ktoré nesúvisia s logikou a ťažko sa vyhodnocujú.
\item Ľudský jazyk je nepresný a nejednoznačný.
\item Potrebujeme nejaké obmedzenia na použité symboly a ich prípustné kombinovanie.
\end{itemize}
\end{frame}

\begin{frame}{Riešenie}
Definovať výroky tak, aby
\begin{itemize}
\item sme sa vyhli pojmu pravdivosti (a pravdivosť výroku definujeme až následne, keď budeme vedieť, čo výrok je);
\pause
\item logická štruktúra výroku (spojky, kvantifikátory) bola oddelená od sveta, ktorý výrok popisuje;
\pause
\item symboly, z ktorých výrok pozostáva, patrili do nejakej fixnej abecedy a ich použitie podliehalo jednoznačným syntaktickým/gramatickým pravidlám.
\end{itemize}
\end{frame}

\begin{frame}{Jazyk logiky}
\alert{Konštanty} zastupujú konkrétne objekty popisovaného sveta.\\[5mm]
\pause
\alert{Predikáty} zastupujú nejaké vlastnosti popisovaného sveta, logike je úplne jedno, aké vlastnosti to sú.
\end{frame}

\begin{frame}{Jazyk logiky: príklad}
\begin{minipage}{0.5\textwidth}
Každý človek umrie.\\[2mm]
Sokrates je človek.\\[2mm]
Sokrates umrie.
\end{minipage}
\begin{minipage}{0.4\textwidth}
$\A x\ \big(C(x)\implies U(x)\big)$\\[2mm]
$C(s)$\\[2mm]
$U(s)$
\end{minipage}
\\[10mm]
Konštanty: s\\
Predikáty: C, U
\end{frame}

\begin{frame}{Jazyk logiky: príklad}
$\A x\in{\mathbb R}$ je skratka pre $\A x\ (x\in {\mathbb R}\implies \dots)$.\\[3mm]
$x\in y$ je skratka pre zápis predikátu: $\in(x, y)$\\[3mm]
Čiže aj teória množín je len špeciálny prípad jazyka logiky.
\end{frame}

\begin{frame}{Jazyk logiky}
\alert{Formula} --- postupnosť symbolov s jasne danými pravidlami (dobre uzátvorkovaná, logické spojky majú správny počet operandov, za kvantifikátorom nasleduje premenná atď.).
\pause
Príklady formúl:
\begin{itemize}
\item $0 = 1$
\item $\A x\ (x \neq 0 \implies x = 1)$
\item $\A x\ \E y\ [P(x) \land Q(x, y)]$
\item $P(x) \implies \lnot Q(7)$\\ --- tzv. otvorená formula, zodpovedá výrokovej forme
\end{itemize}
\end{frame}

\begin{frame}{Jazyk logiky}
Jazykov logiky existuje viacero, líšia sa
\begin{itemize}
\item konštantami
\item predikátmi
\item povolenými logickými spojkami (napr. zakážeme ekvivalenciu)
\item prítomnosťou kvantifikátorov
\item \dots
\end{itemize}
Niektoré obmedzenia (napr. kvantifikátory) majú zásadný vplyv na to, čo možno v jazyku vyjadriť, a následne na algoritmické ťažkosti pri určovaní vlastností formúl.
\end{frame}

\begin{frame}{Pravdivosť formúl}
Je $P(\hbox{alf})$ pravda?\\[5mm]
Aby bolo možné skúmať pravdivosť, treba pridať \alert{interpretáciu} pre všetky predikáty a konštanty (\emph{non-logical} symbols).
\end{frame}

\begin{frame}{Pravdivosť formúl: interpretácia}
\begin{itemize}
\item Konštanty budú prvky nejakej množiny $D$ (doména).
\pause
\item Predikát je pravdivý pre niektoré prvky z $D$ (podmnožina domény).
\pause
\item Kvantifikované premenné nadobúdajú hodnoty z domény:
	$\E x\ P(x)$ vyjadruje, že v $D$ sa nachádza prvok $d$, pre ktorý platí $P(d)$.
\end{itemize}
\pause
\vspace*{3mm}
Úloha: uvažujme formulu $\A x\ (x = c_0 \lor x = c_1)$.\\
Nájdite doménu a interpretáciu konštánt, pre ktoré je táto formula
(a) pravdivá, (b) nepravdivá.
\end{frame}

\begin{frame}{Načo je to dobré? Databázy}
Databáza: tabuľky, v ktorých sú uložené zoznamy objektov.
\begin{table}[h!]
    \centering
	\begin{tabular}{|l|l|}
        \hline
        \emph{Meno} & \emph{Rodné číslo} \\
        \hline
        Ján & 1234 \\
        Petra & 5678 \\
        Michal & 9876 \\
        \hline
    \end{tabular}
\end{table}
Táto tabuľka popisuje akýsi predikát $P$ s dvomi argumentmi.\\
Dátové typy určujú doménu, naplnenie tabuliek dátami určuje interpretáciu.\\[3mm]
Databázy sa potom pýtame:\\[2mm]
\uv{rodné čísla ľudí, ktorí sa volajú Jozef} $ = \{rc\ |\ P(Jozef, rc)\}$
\uv{rodné čísla všetkých ľudí}\hspace*{21.5mm} $ = \{rc\ |\ \E m\ P(m, rc)\}$
\end{frame}

\begin{frame}{Načo je to dobré? Databázy}
Databázový systém určuje, ktoré formuly sú povolené.\\
Čím zložitejšie formuly, tým viac možno vyjadriť, ale ťažšie sa vyhodnocujú.\\[3mm]
Aj preto sa dotazy píšu v špeciálnych jazykoch ako SQL, nie v C či Pythone.
\end{frame}

\begin{frame}{Logické vyplývanie}
Jedna formula môže byť pravdivá i nepravdivá, závisí od interpretácie predikátov a konštánt.

Zväčša nás nezaujíma vymýšľanie interpretácií, ale vzťahy medzi formulami
--- logické vyplývanie. Napr. či z pravidiel aritmetiky, popísaných formulou $PA$, vyplýva tvrdenie $F$: \uv{súčet dvoch prvočísel je vždy párny}.\\[2mm]

Toto sa dá úplne presne definovať:
\alert{Je pravda, že v každej interpretácii, kde je $PA$ pravdivá, je aj $F$ pravdivá?}
\end{frame}

\begin{frame}{Logické vyplývanie}
Ak máme konečnú doménu, je ľahké rozhodnúť o vyplývaní:
\begin{itemize}
\item vieme vyhodnotiť kvantifikátory, lebo za premennú možno dosadiť len konečne veľa objektov z domény;
\item je len konečne veľa možných interpretácií predikátov a stačí všetky vyskúšať.
\end{itemize}
Lenže typicky máme doménu nekonečnú (napr. reálne čísla) a kvantifikátorov sa nevieme zbaviť, lebo by sme stratili vyjadrovaciu silu jazyka. Preto miesto skúmania pravdivosti robíme dôkazy.
\end{frame}

\begin{frame}{Čo je to dôkaz?}
Kľúčová idea: dokazovacie systémy sú mechanické, algoritmické,
realizovateľné na počítači bez úvah o interpretáciách a pravdivosti.
\end{frame}

\begin{frame}{Čo je to dôkaz?}
Príklad dokazovacieho systému: \alert{priamy dôkaz}. Je to postupnosť formúl, z ktorých každá je axiómou, alebo vzniká aplikáciou pravidla na predchádzajúce formuly.
Príklady pravidiel:
\begin{itemize}
\item ak v postupnosti máme $A\implies B$ aj $A$, tak môžeme pripísať $B$.
\item ak v postupnosti máme $A\implies B$ aj $\lnot B$, tak môžeme pripísať $\lnot A$.
\end{itemize}

Pravidlá musia byť \alert{korektné}: pripísaná formula musí logicky vyplývať z predošlých. To zaručí, že čokoľvek, čo dokážeme, je pravda.\\[3mm]

Pravidlá musia byť \alert{syntaktické}: pri aplikácii pravidla sa riadime čisto štruktúrou formuly.
\end{frame}

\begin{frame}
Hilbertov program (1900):
Nájdime takú sadu axióm (teóriu), že pre každú matem. formulu $F$ možno dokázať $F$ alebo $\lnot F$.\\[5mm]
(Zovšeobecnenie Euklidových Základov.)
\end{frame}

\begin{frame}{G\"odelova veta o neúplnosti (1930)}
Nemožno nájsť systém axióm $T$, ktorý má zároveň nasledujúce vlastnosti:
\begin{itemize}
\item $T$ obsahuje aritmetiku (sčítanie, násobenie, prirodzené čísla),
\item $T$ je efektívne axiomatizovateľná [ak by nie, nevieme algoritmicky ani len rozhodnúť, čo je formula či dôkaz],
\item z $T$ nevyplýva nepravda [ak by vyplývala, tak z tejto nepravdy už vyplýva čokoľvek, čiže $T$ je nanič],
\item $T$ je korektný [čo sa dá dokázať, je pravda],
\item $T$ je negačne úplný (pre každú formulu vyjadriteľnú v jazyku $T$ máme jej dôkaz alebo vyvrátenie).
\end{itemize}
\end{frame}
% Idea dôkazu: Godelovo tvrdenie G = "toto je veta, ktorá sa nedá dokázať".

\begin{frame}{Ďalšie nepríjemnosti}
Ak povolíme kvantifikátory, \alert{neexistuje algoritmus}, ktorý by pre danú formulu rozhodol, či je pravdivá.

Inak: akýkoľvek postup na rozhodovanie pravdivosti pre nejakú vstupnú formulu neskončí (alebo dá nesprávny výsledok).
\end{frame}

\begin{frame}{Čo robia automatizované dokazovače?}
Jazyk logiky sa dá navrhnúť tak, že pravdivosť je totožná s dokázateľnosťou.\\[3mm]

Dokazovač potom skúsi postupne generovať všetky možné dôkazy (napr. od najkratších), a ak nájde dôkaz danej formuly, vie, že je pravdivá.
(Keby sme postupne skúšali všetky domény a interpretácie, nikdy nemôžeme povedať, že formula je pravdivá.)\\[3mm]

V praxi je tento postup prekvapivo úspešný (používajú sa iné metódy, nie priame dôkazy).
\end{frame}

\begin{frame}{Symbolické vs. numerické výpočty}
\begin{minipage}[t]{0.45\textwidth}
Symbolický výpočet:
\begin{align*}
	x^2 &= 2\\
	(x+\sqrt 2)(x-\sqrt 2) &= 0\\
	x &= \pm\sqrt 2
\end{align*}
Symboly majú jasný význam, výpočet pozostáva z overiteľných krokov, ktoré samé osebe \uv{dávajú zmysel}.
\end{minipage}
\quad
\begin{minipage}[t]{0.45\textwidth}
Numerický výpočet:
\begin{align*}
	x^2 &= 2\\
	x &\in (1, 2)\\
	x &\in (1.4, 1.5)\\
	  &\dots\\
	x &\approx 1.4142
\end{align*}
Kroky výpočtu nenesú samé osebe zmysel, sú to len aritmetické operácie, výsledok je nespoľahlivý.
\end{minipage}
\end{frame}

\begin{frame}{Symbolické vs. aproximačné výpočty}
\begin{minipage}[t]{0.45\textwidth}
Symbolické:
\begin{itemize}
	\item úprava výrazov
	\item matematické dôkazy
	\item expertné systémy (kľúč na určovanie druhu húb)
\end{itemize}
\end{minipage}
\quad
\begin{minipage}[t]{0.45\textwidth}
Aproximačné / data-driven:
\begin{itemize}
	\item numerická optimalizácia
	\item strojové učenie
	\item LLM (ChatGPT)
\end{itemize}
\end{minipage}
\end{frame}

\begin{frame}{Symbolické vs. aproximačné výpočty}
Nevýhodou výpočtov založených na dátach je chýbajúca kontrola nad smerovaním výpočtu a nemožnosť pochopenia / overenia.

Napr. ChatGPT generuje text, ktorý je \uv{pravdepodobný} (vzhľadom na texty v trénovacích vstupoch). Nijako nevie merať ani overovať správnosť. Preto na začiatku nezvládal ani sčítanie jednociferných čísel. Natrénovaný model umelej inteligencie je niekoľko miliárd reálnych čísel usporiadaných do tabuľky. Z pohľadu naň nik netuší, čo počíta, a sám model nemá žiaden mechanizmus na vysvetlenie toho, čo sa deje.
\end{frame}

\begin{frame}{AGI: kombinácia symbolických a aproximačných výpočtov}
Ak chceme závery, pri ktorých možno garantovať správnosť, potrebujeme symbolický výpočet. Ten je však ťažké spraviť: treba \uv{uhádnuť}, ako má vyzerať dôkaz. Potom už ľahko overíme správnosť.\\[3mm]

Na uhádnutie použijeme generatívnu AI:\\
AlphaProof: navrhuje, ako by mohli vyzerať kroky výpočtu.\\
AlphaGeometry: navrhuje, aké body, úsečky či kružnice dokresliť do zadania.
\end{frame}

\end{document}




