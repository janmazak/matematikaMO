\documentclass[12pt]{beamer}
\usetheme{default}
\usecolortheme{crane}
%\usetheme{Madrid}

\usepackage[utf8x]{inputenc}
\usepackage[T1]{fontenc}
\usepackage[slovak]{babel}
\usepackage{ucs}
\usepackage{amsmath}
\usepackage{graphicx}
\usepackage{array}
\usepackage{amsmath, amssymb}
%\usepackage[inline]{asymptote}

%\setbeamersize{text margin left=1pt,text margin right=1pt}

\theoremstyle{definition}
\newtheorem{uloha}{Úloha}

\let\o=\vee
\let\a=\wedge
\let\bigo=\bigvee
\let\biga=\bigwedge
\let\A\forall
\let\E\exists

\begin{document}
	\title{Ako moderná logika vyriešila problémy s~pravdivosťou výrokov}
	\author{Ján Mazák}
	\frame{\titlepage}
	\institute{
		\inst{1}
		UK Bratislava}

\begin{frame}
%\frametitle{Čo je výrok?}
Čo je výrok? \emph{Oznamovacia veta},
\begin{itemize}
	\item ktorá má pravdivostnú hodnotu.
	\item ktorej sa dá priradiť pravdivostná hodnota.
	\item pre ktorú má zmysel otázka na jej platnosť, správnosť, pravdivosť.
\end{itemize}
\end{frame}

\begin{frame}
Nech $x$ je kladné reálne číslo. Potom $x^2 > 0$.
\end{frame}

\begin{frame}
Predstavme si, že $p$, $q$ sú výroky.
\begin{itemize}
	\item Je \uv{$p$ or not $p$} výrok?
	\item Je \uv{$p$ or not $q$} výrok?
\end{itemize}
\end{frame}

\begin{frame}
Ako definovať \emph{jednoduchý výrok}? \uv{Neobsahuje logické spojky.}
\begin{itemize}
	\item $x$ delí $y$
	\item existuje číslo $k$ také, že $k$ je prirodzené a $y = kx$
\end{itemize}
Hovoríme to isté, lenže raz s logickou spojkou a raz bez.\\

Pritom ak chceme uchopiť zložitosť výrokov, treba aspoň vedieť určiť, ktoré sú tie najjednoduchšie.
\end{frame}

\begin{frame}
Mimo Zeme žijú niekdajší obyvatelia Atlantídy.\\[5mm]
Počet hviezd je nepárny.
\end{frame}

\begin{frame}
Zajtra bude pršať.\\[5mm]
\pause
Odteraz až navždy bude každý deň pršať.
\end{frame}

\begin{frame}
Každé párne číslo je súčtom dvoch prvočísel.
\\(Goldbachova hypotéza)
\end{frame}

\begin{frame}
Postupnosť $0,1,2,3,4,5,6,\dots$ je rastúca.
\begin{itemize}
\item Je to výrok?
\pause
\item Ako vieme, ako bude postupnosť pokračovať?
\pause
\item Aký je význam troch bodiek ako symbolu? Môže byť súčasťou výroku popis algoritmu?
\pause
\item Môže byť výrok nekonečne dlhý?
\end{itemize}
\end{frame}

\begin{frame}
Bu bayonot emas.
\begin{itemize}
\item Znamená \uv{toto nie je výrok} v uzbečtine. Aký jazyk je prípustný?
\pause
\item Môže výrok hovoriť o sebe?
\item Môže viesť k paradoxom: Zoberme množinu $X$ všetkých množín, ktoré neobsahujú samé seba.
Je veta \uv{$X$ patrí do $X$} výrok? Nemôže to byť ani pravda, ani nepravda,
pritom to vyzerá ako úplne korektné matematické vyjadrenie.
\end{itemize}
\end{frame}

\begin{frame}{Sumarizácia problémov}
\begin{itemize}
\item Pravdivosť je veľmi ťažké uchopiť.
\item Výrok v sebe nesie odkazy na reálny svet, ktoré nesúvisia s logikou a ťažko sa vyhodnocujú.
\item Potrebujeme nejaké obmedzenia na použité symboly a ich prípustné kombinovanie.
\end{itemize}
\end{frame}

\begin{frame}{Riešenie}
Definovať výroky tak, aby
\begin{itemize}
\item sme sa vyhli pojmu pravdivosti (a pravdivosť výroku definujeme až následne, keď budeme vedieť, čo výrok je);
\pause
\item logická štruktúra výroku (spojky, kvantifikátory) bola oddelená od sveta, ktorý výrok popisuje;
\pause
\item symboly, z ktorých výrok pozostáva, patrili do nejakej fixnej abecedy a ich použitie podliehalo jasným syntaktickým/gramatickým pravidlám.
\end{itemize}
\end{frame}

\begin{frame}{Jazyk logiky}
\alert{Konštanty} zastupujú konkrétne objekty popisovaného sveta.\\[5mm]
\pause
\alert{Predikáty} zastupujú nejaké vlastnosti popisovaného sveta, logike je úplne jedno, aké vlastnosti to sú.
\end{frame}

\begin{frame}{Jazyk logiky}
\alert{Formula} --- postupnosť symbolov s jasne danými pravidlami (dobre uzátvorkovaná, logické spojky majú správny počet operandov, za kvantifikátorom nasleduje premenná atď.).
\pause
Príklady formúl:
\begin{itemize}
	\item $0 = 1$
	\item $\A x (x \neq 0 \implies x = 1)$
	\item $\A x \E y [P(x) \land Q(x, y)]$
	\item $P(x) \implies \lnot Q(7)$\\ --- tzv. otvorená formula, zodpovedá výrokovej forme
\end{itemize}
\end{frame}

\begin{frame}
Logika zrazu nie je len jedna --- jednotlivé logické jazyky sa líšia o.i. v konštantách, môžeme tiež obmedziť použitie logických spojok.
\end{frame}

\begin{frame}
Chceme vedieť algoritmicky rozhodovať o tom, ktorá postupnosť symbolov je formula?
\end{frame}

\begin{frame}{Pravdivosť formúl}
Je $P(\hbox{alf})$ pravda?\\[5mm]
Aby bolo možné skúmať pravdivosť, treba pridať \alert{intepretácie} pre všetky mimologické symboly.
\end{frame}

\begin{frame}{Pravdivosť formúl}
\begin{itemize}
\item Konštanty budú prvky nejakej množiny $D$ (doména).
\pause
\item Predikát je pravdivý pre niektoré prvky z $D$ (podmnožina domény).
\pause
\item Kvantifikované premenné nadobúdajú hodnoty z domény:
	$\E x P(x)$ vyjadruje, že v $D$ sa nachádza prvok $d$, pre ktorý platí $P(d)$.
\end{itemize}
\pause
\vspace*{3mm}
Úloha: uvažujme formulu $\A x (x = c_0 \lor x = c_1)$.\\
Nájdite doménu a interpretáciu konštánt, pre ktoré je táto formula
(a) pravdivá, (b) nepravdivá.
\end{frame}

\begin{frame}{Načo je to dobré? Databázy}
Bežná databáza obsahuje tabuľky, v ktorých sú zoznamy objektov spĺňajúcich predikáty.
Napr. tabuľka s dvojicami (meno, rodné číslo) popisuje predikát s dvomi argumentmi.
Databázy sa potom pýtame otázky v podobe \uv{nájdi mi všetky $x$, ktoré spĺňajú nejakú formulu vyjadrujúcu požadovanú vlastnosť $x$}.
Napr. \uv{rodné čísla ľudí, ktorí sa volajú Jozef}.
\end{frame}

\begin{frame}{Načo je to dobré? Databázy}
Pri návrhu databázového systému potrebujeme presne popísať, aké formuly je povolené použiť v dotazoch.
Čím zložitejšie formuly, tým viac nimi možno vyjadriť, ale zároveň sa ťažšie vyhodnocujú.
Aj preto sa dotazy píšu v špeciálnych jazykoch ako SQL, nie v C či Pythone.
\end{frame}


\begin{frame}{Pravdivosť formúl}
Podarilo sa nám oddeliť vlastnosť \uv{byť formulou} od pravdivosti.
Jedna formula môže byť pravdivá i nepravdivá, závisí od interpretácie mimologických symbolov.\\[3mm]

Množine formúl hovoríme teória. Teória tiež môže byť pravdivá i nepravdivá.
Interpretáciám, v ktorých je teória pravdivá, hovoríme model.
\end{frame}

\begin{frame}{Pravdivosť formúl}
Zväčša nás nezaujíma vymýšľanie interpretácií, ale vzťahy medzi formulami
--- logické vyplývanie.
Typická otázka je, či z teórie $T$ vyplýva formula $F$.
Napr. keď ako teóriu $T$ vezmeme aritmetiku a pýtame sa, či z $T$ vyplýva \uv{súčet dvoch prvočísel je vždy párny}.\\[2mm]

Toto sa dá úplne presne definovať: je pravda, že každý model $T$ je modelom $F$?\\
\alert{Je pravda, že v každom svete, v ktorom sú splnené všetky formuly z $T$, je splnená aj $F$?}
\end{frame}

\begin{frame}
Úloha: Nech $A$, $B$ sú formuly.\\
Je pravda, že z $A \lor\lnot B$ a $B$ vyplýva $A$?
\end{frame}

\begin{frame}{Čo je to dôkaz?}
Kľúčová idea: dokazovacie systémy sú mechanické, algoritmické,
realizovateľné na počítači bez akýchkoľvek úvah o tom,
čo je pravda či pre ktoré interpretácie je formula pravdivá.

Korektnosť a úplnosť.
\end{frame}

\begin{frame}{Čo je to dôkaz?}
Kľúčová idea: dokazovacie systémy sú mechanické, algoritmické,
realizovateľné na počítači bez akýchkoľvek úvah o tom,
čo je pravda či pre ktoré interpretácie je formula pravdivá.
\end{frame}

\begin{frame}
Hilbertov program (1900):
nájdime takú sadu axióm (teóriu), že pre každú matem. formulu $F$ možno dokázať $F$ alebo $\lnot F$.\\[5mm]
Čosi podobne sa podarilo Euklidovi s planimetriou: na základe 5 axióm (v skutočnosti asi 20)
dokázal platnosť všetkých podstatných planimetrickych tvrdení známych Grékom.
\end{frame}

\begin{frame}{G\"odelova veta o neúplnosti (1930)}
Nemožno axiomatizovať ani len aritmetiku.
Presnejšie, nemožno nájsť teóriu $T$, ktorá má zároveň nasledujúce vlastnosti:
\begin{itemize}
\item $T$ je prvého rádu a je korektná [ak nie, pravdivosť nebude totožná s dokázateľnosťou],
\item z $T$ nevyplýva nepravda [ak by vyplývala, tak z tejto nepravdy už vyplýva čokoľvek, čiže $T$ je nanič],
\item $T$ je efektívne axiomatizovateľná [ak by nie, nevieme algoritmicky ani len rozhodnúť, čo je formula či dôkaz],
\item $T$ obsahuje aritmetiku (sčítanie, násobenie, prirodzené čísla),
\item $T$ je negačne úplná (pre každú formulu vyjadriteľnú v jazyku $T$ máme jej dôkaz alebo vyvrátenie).
\end{itemize}
\end{frame}
% Idea dôkazu: Godelovo tvrdenie G = "toto je veta, ktorá sa nedá dokázať".

\begin{frame}{Čomu sa venuje logika dnes?}
\begin{itemize}
\item zúženia prvorádovej logiky (napr. monadická), logika vyššieho rádu
\item rôzne výpočtové aspekty (automatické dokazovače, SAT solvery, zložitosť problémov)
\item fuzzy logika, modálne logiky (reprezentácia \uv{možno}/\uv{morálne akceptovateľné}/\dots)
\item mnoho teoretických tém, ktorým nerozumiem
\end{itemize}
\end{frame}


\end{document}




